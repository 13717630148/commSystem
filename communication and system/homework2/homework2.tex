\documentclass{article}
\usepackage{CJKutf8}
\usepackage{multirow}
\usepackage{listings}
\usepackage{graphicx}

\begin{CJK}{UTF8}{gbsn}
\begin{document}
\title{通信系统作业一:电缆通信}
\author{王亭午, 无210班, 2012011018}
\date{2015年4月11号}
\maketitle
\section{作业一}
请查阅千兆以太网采用的传输机制,与百兆以太网的传输机制比较,说明从百兆到千兆的速率变化的原因?\\
答案:千兆以太网和百兆以太网在传输机制上并没有太多不同,都采用CSMA/CD方法,但是存在一下几个区别。\\

\indent 1. 千兆以太网的传输带宽更大,传输速率更大。\\
\indent 2. 千兆以太网成本更加高昂。\\
\indent 3. 千兆以太网配置更加复杂,需要特殊的传输介质。\\

实际上千兆以太网在百兆以太网提出后不到5年就被人提出来,而当时无法与百兆以太网竞争的原因在于:\\
\indent 1. 成本过高。\\
\indent 2. 计算机处理速度较慢,千兆以太网没有意义。\\
随着六类网线等高速传输介质成本的下降,电脑运算速度的快速提升以及千兆网卡技术的提升,千兆网因为其平滑替代百兆网的特性,快速发展中。
\end{CJK}
\end{document}
 802.3z
 IEEE 802.3u