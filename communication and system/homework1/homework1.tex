\documentclass{article}
\usepackage{CJKutf8}
\usepackage{multirow}
\usepackage{listings}
\usepackage{graphicx}

\begin{CJK}{UTF8}{gbsn}
\begin{document}
\title{通信系统作业一:电缆通信}
\author{王亭午, 无210班, 2012011018}
\date{2015年3月29号}
\maketitle
\section{作业一}
紫荆公寓某宿舍,购买了数字化语音的无线子母机,用于拨打固定电话;在安装时,将母机用双绞线连接到宿舍墙上的电话接口,将无线子机放在中厅;当同学使用无线子机拨打家里的固定电话时,这时建立的通话双方的传输链路上,包括哪些设备单元?各相邻设备单元之间的传输的是数字信号还是模拟信号?\\
\indent 答案:当同学使用无线子机进行呼叫的时候,经历了以下几个过程。
子机呼叫,母机响应子机建立无线连接,主叫拨被叫号码,被叫应答,开始通话,话毕挂机。\\
由此在传输链路上,有以下几个设备单元。从家中固定电话到同学子机则是相反的过程。
\begin{table}[!hc]
\begin{tabular}{|p{2cm}|p{4cm}|p{2cm}|p{2.3cm}|}
\hline
信号发送点&信号处理&传输形式&信号接受点\\
\hline
子机&采样人声为数字信号&无线数字信号&母机\\
\hline
母机&和上行数据一起加载为模拟信号(e.g.:ADSL)&模拟信号,双绞线&第一级交换局\\
\hline
低级交换局&通过交换机(时分交换网络),中继线&数字信号&高级交换局\\\hline
高级交换局&通过交换机(时分交换网络),中继线&数字信号&低级交换局\\\hline
低级交换局&和下行数据一起加载为模拟信号(e.g.:ADSL)&模拟信号,双绞线&家中固定电话\\
\hline
\end{tabular}
\caption{信号从同学子机到家里固定电话经过的设备单元} %表格的名称
\end{table}
\end{CJK}
\end{document}\\