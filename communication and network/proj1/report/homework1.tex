\documentclass{article}
\usepackage{CJKutf8}
\usepackage{multirow}
\usepackage[cmex10]{amsmath}
\usepackage{listings}
\usepackage{algorithmic}
\usepackage{algorithm}
\usepackage{epsfig}
\usepackage{amsopn}
\usepackage{subfigure}
\usepackage{cite}
\usepackage{courier}
\makeatletter
\makeatother
\begin{CJK}{UTF8}{gbsn}
\usepackage[framed,autolinebreaks,useliterate]{mcode}

\usepackage{graphicx}
\begin{document}
\title{通信与网络\\第一次编程作业}
\author{王亭午,2012011018,无210班}
\date{2015年10月8号}
\maketitle
\section{任务一}
某信道,输入为M元逻辑符号x:\(s_0,s_1,\cdots,s_{M-1}\)。
输出y为实数值信道中发生如下事件:
\(a=f(x)\)到实数的一一映射,当\(x=s_i\)时,\(a=iA\),A为一给定的正实数,
\(y=a+n\),n为一服从\(N(0,s_2)\)分布的独立随机变量(与x独立,且每次信道实现时的n均独立)。\\
写出信道转移概率。若输入信道的各符号等概出现,求该信道的互信息量I(X;Y)。画出不同信噪比下的互信息量变化的曲线,以M为参数,画一簇曲线(其中加上一条AWGN信道容量曲线作对比)。
调整函数\(a=f(x)\),使当\(x=s_i\)时,\(a=iA-b\),b也为一实常数,在A和s不变的情况下,互信息量随b的变化情况是什么趋势?
b的取值对互信息量随信噪比的变化曲线的影响?
\subsection{信道转移概率和互信息量}
答案:由题可知,这个转移概率是一个简单的条件高斯概率分部,如下
\begin{equation}
p(y|x = s_i) = \frac{1}{\sqrt{2\pi\sigma^2}} e^{-\frac{(x-iA)^2}{2\sigma^2}}
\end{equation}
其中\(\sigma^2 = s_2\)。\\
互信息量的求解如下
\begin{equation}
\begin{aligned}
I(X;Y) & = H(Y) - H(Y|X)\\
	   & = -\int_{p(y) \neq 0, -\infty}^{\infty}p(y)\mbox{log}p(y)dy - \frac{1}{2}\mbox{log}(1 + \frac{P}{N})\\
	   & = -\int_{p(y) \neq 0, -\infty}^{\infty}p(y)\mbox{log}p(y)dy - \frac{1}{2}\mbox{log}(1 + s_2)
\end{aligned}
\end{equation}
其中\(p(y)\)的求解如下
\begin{equation}
\begin{aligned}
p(y) &= \frac{1}{M}\sum\limits_{i = 0}^{M}p(y|x = s_i)\\
	 &= \frac{1}{M}\sum\limits_{i = 0}^{M}\frac{1}{\sqrt{2\pi\sigma^2}} e^{-\frac{(x-iA)^2}{2\sigma^2}}
\end{aligned}
\end{equation}
从上面的式子(1),(2),我们可以知道,上面的mutual information可以通过数值计算的方法得到,代码如下:
\begin{lstlisting}
for i_M = 1: 1: length(M)
    for i_variation = 1: 1: length(sigma2)
        entropy_y(i_M, i_variation) = ...
            integral(@(x)entropy_type1(x, M(i_M), sigma2(i_variation)), ...
            0 - 3 * sigma2(i_variation), 3 * sigma2(i_variation) + M(i_M));
        entropy_y(i_M, i_variation) = ...
            entropy_y(i_M, i_variation) - ...
            1 / 2 * log2(2 * pi * exp(1) * sigma2(i_variation)^2);
        fprintf('i_M = %f, s2 = %f\n', i_M, sigma2(i_variation))
    end
    fprintf('The %d th line finished\n', i_M)
end
\end{lstlisting}
在上面的方法中,我们对不同的M和不同的噪声进行了求解。我们使用integral函数进行数值上的积分求解。
在数值的设定上,我们假设A=1,如下
\begin{lstlisting}
% In the function, the A is set as "1" for convenience and the noise is set
% as "sigma2" as a relative parameters, note here the s2 is the standard
% variantion, instead of its square
A = 1;
sigma2 = 0:0.1:30;
M = 1:10;
\end{lstlisting}
sigma2则是我们的\(\sigma_2\),代表噪声的变化,单位上也和A的单位进行了归一化。
上面函数中的最为关键的entropy\_type1函数如下,这个计算的是连续事件的信息量。
\begin{lstlisting}
function entropy = entropy_type1(x, M, sigma)
%The information of single event, both continous and discrete.
%
% Function of this function is to calculate the entropy of a certain 
% variable, which might not have a expression and only numerical resutls
% are available
%
% Written by Tingwu Wang, Tsinghua University, for the course project
% 27/09/2015
prob = 0;
sym_prob = 1 / M;
coff1 = 1 / (sqrt(2 * pi) * sigma);
coff2 = 2 * sigma .^2;
for i_symbol = 0: 1: M - 1
    %prob = prob + sym_prob * normpdf(x, i_symbol, sigma);
    prob = prob + sym_prob * coff1 * exp(-(x - i_symbol).^2 / coff2);
end
entropy = entropy_cal(prob);
\end{lstlisting}
其中的entropy\_cal是一个简单的计算xlog(x)的函数。
值得注意的是,我们比较了使用系统自带的normpdf函数和我们手动写一个计算概率的情况,发现如果我们自己重写,速度是快很多的。
\subsection{结果图}
经过上面的算法,我们已经可以计算出我们的结果了。画图代码如下:
\begin{lstlisting}
% plot the results
legend_name = cell(length(M) + 1, 1);
cmap = hsv(length(M) + 1);  % Creates a 6-by-3 set of colors from the HSV colormap
for i_M = 1: 1: length(M)
  plot(sigma2, entropy_y(i_M, :), 'Color', cmap(i_M, :));
  hold on;
  box on;
  grid on;
  legend_name{i_M} = ['M = ' num2str(i_M)];
end

% awgn channel
awgn_entropy = zeros(1, length(sigma2));

for i_variation = 1: 1: length(sigma2)
    awgn_entropy(i_variation) = 1 / 2 * log2(1 + 1 / sigma2(i_variation));
end
legend_name{length(M) + 1} = 'AWGN';
plot(sigma2, awgn_entropy(:), 'Color', cmap(length(M) + 1, :));
legend(legend_name)
\end{lstlisting}
结果如下:
\begin{figure}	
\begin{center}
		\includegraphics[width=1\linewidth]{capacity.eps}
		\caption{信道容量示意图}
\end{center}
\end{figure}
\subsubsection{结果分析}
从结果来看,我们可以得出以下几个结论。\\
1.随着噪声的增加,我们的capacity不断减少。不过始终是大于0的,这和我们的预测相符合。\\
2.随着我们使用的信源数量增加,即M不断增加,我们可以达到的预期capacity也是会增加的,这符合我们的预期。\\
3.可以看到,当噪声比较小的情况下,我们的所求得的capacity小于我们的AWGN,但是当噪声较大的时候,AWGN信道反而差于我们的情况。当然,我们应该注意,千万不要把两种情况作出不合适的对比。因为我们知道,AWGN只有在我们的输入信号是功率受限的时候才是最优化的。所以出现这种信道小于我们特殊情况的信道是正常的。\\
4.在其中我们设置了一个验证,当M=1的时候,我们的信道不具备任何变化,因此信道应该为0。这和我们的结果吻合。
5.在图2中,我们可以看到,随着我们的M不断增加,信道容量不断增加,而且没有趋于饱和迹象。
\begin{figure}	
\begin{center}
		\includegraphics[width=1\linewidth]{capacity_2.eps}
		\caption{信道容量示意图,M=1:1:100}
\end{center}
\end{figure}
\subsection{信道容量随信号平移的变化}
观察我们的新的信道,有下列式子成立
\begin{equation}
\begin{aligned}
I(X;\bar{Y}|b) & = H(\bar{Y}) - H(\bar{Y}|X)\\
	   & = -\int_{p(\bar{y}) \neq 0, -\infty}^{\infty}p(\bar{y})\mbox{log}p(\bar{y})d\bar{y} -\int_{p(\bar{y}|x) \neq 0, -\infty}^{\infty}p(\bar{y}|x)\mbox{log}p(\bar{y}|x)d\bar{y}\\
	   & \overset{a}= -\int_{p(y) \neq 0, -\infty}^{\infty}p(y)\mbox{log}p(y)dy - \int_{p(y|x) \neq 0, -\infty}^{\infty}p(y|x)\mbox{log}p(y|x)dy\\
	   & = I(X;Y),\\
	   &\text{where a means:  } y = \bar{y} + b
\end{aligned}
\end{equation}
可见我们的信道容量和我们的b无关。
\end{CJK}
\end{document}